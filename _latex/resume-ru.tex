\documentclass{yb}
\usepackage[T2A,T1]{fontenc}
\usepackage[utf8]{inputenc}
\usepackage[english,main=russian]{babel}

\begin{document}

\ybPrintPhoto{}

{\bfseries\Large \nospell{Егор Георгиевич Буга\'{е}нко}}\newline
\href{mailto:yegor256@gmail.com}{yegor256@gmail.com}

\vspace{1em}

Директор лаборатории в \textbf{\href{https://www.huawei.com}{Huawei}} с 2019-го года.

Основатель платформы \textbf{\href{https://www.zerocracy.com}{Zerocracy}} с 2016-го.

Создатель eXtremely Distributed Software Development (\href{https://www.xdsd.org}{XDSD}).

В управлении проектами с 1998-го (в семи компаниях).

Автор книг \textbf{\href{https://www.yegor256.com/elegant-objects.html}{Elegant Objects}}
  об объектно-ориентированном программировании,
  и \href{https://www.yegor256.com/books.html}{нескольких} других.

Технический \textbf{блогер} на \href{https://www.yegor256.com/}{yegor256.com} с 2014-го.

Создатель не-Blockchain криптовалюты \href{https://www.zold.io}{Zold}.

Организатор ACM/IEEE научной конференции \href{https://www.iccq.ru}{\textbf{ICCQ}}.

Обладатель сертификатов \textbf{SCEA} и \textbf{PMP}.

Создатель \href{http://www.takes.org}{Takes},
  \href{http://www.cactoos.org}{Cactoos},
  \href{http://www.jcabi.com}{JCabi},
  \href{http://www.rultor.com}{Rultor},
  \href{http://www.qulice.com}{Qulice}
  и \href{https://www.yegor256.com/pets.html}{других} продуктов с открытым кодом.

\textbf{Магистр} по специальности ``Вычистительные Машины'',
  \href{https://en.wikipedia.org/wiki/Oles_Honchar_Dnipro_National_University}{выпуска} 1998-го.

Частый \textbf{\href{https://www.yegor256.com/talks.html}{докладчик}}
  на таких конференциях, как
  \href{https://youtu.be/55mwAbuDrV8}{Joker},
  \href{https://www.youtube.com/watch?v=20QBvrHq6TA}{JPoint},
  \href{https://vimeo.com/177215750}{GeeCON},
  \href{https://www.youtube.com/watch?v=TLM9eN0b6zo}{AgileDays},
  \href{https://www.youtube.com/watch?v=03PXmPc7Q3g}{\nospell{{\O}redev}},
  \href{https://www.youtube.com/watch?v=7yTIWFZrXpg}{GeekOUT},
  \href{https://www.youtube.com/watch?v=vU_x6oK437I}{NDC},
  \href{https://www.youtube.com/watch?v=GS45LzE3LPQ}{JEEConf}.

Профессиональный член ACM и IEEE с 1996-го.

Программист на Java/C++/Ruby
  (3.6K+ \textbf{\href{https://github.com/yegor256}{GitHub}} подписчиков).

Активный участник \textbf{Stack Overflow} сообщества
  (\href{https://stackexchange.com/users/63162/yegor256}{100K+ профиль}).

Автор нескольких патентных заявков в США
  (например, \href{https://www.google.com/patents/US20120023476}{12/840,306}).

Автор нескольких научных \textbf{публикаций}
  (например, \href{https://link.springer.com/chapter/10.1007/978-3-642-02152-7_6}{этой}).

Один из со-авторов стандарта \href{http://standards.ieee.org/develop/wg/730.html}{IEEE 730}.

Поклонник
  тенниса,
  киокушинкай,
  \href{https://www.yegor256.com/paintings.html}{искусства},
  и
  \href{https://ru.yegor256.com}{политики}.

Активный участник
  \href{https://twitter.com/intent/follow?screen_name=yegor256}{Twitter} (15K+),
  \href{https://www.facebook.com/yegor256}{Facebook} (4K+),
  и
  \href{https://instagram.com/yegor256}{Instagram} (5K+).

\end{document}
