\documentclass{./yb}
\newcommand\first[2]{\vspace{1em}{\setlength{\parindent}{0pt}\bfseries\LARGE #1}\textsc{#2}}

\begin{document}

{\bfseries\scshape Teaching Statement}\newline
Yegor Bugayenko\newline
\today

\vspace*{12pt}

\first{I}{ don't} believe in teaching. I believe in learning, which
just happens when someone has a motivation and an ability.
Well, I think I'm wrong about the ability---just motivation!
Once there is a strong \textbf{motive}, the student is at the
driver's seat, while teachers, mentors, books, and video
courses---are his or her mere assistants, at best.

A motive is the derivative of the answer, which so many students
struggle to find for their most important simple
question: \emph{``What's in it for me?''}
Indeed, why do they need to study
the relational theory, distributed transaction processing,
Boolean algebra, or $\lambda$-calculus?

``Because you need a diploma,'' is the easiest response some
teachers resort to. I remember myself being a student 25 years ago---such
an answer was not at all convincing for me. No matter how harsh
were the exams, how strict was the discipline, and how much did I pay
for the study---I needed something else. And I had it.

Obviously, every student has its own motive.
Quite often it's a very unique one.
Some of them dream about their own
career in science and the answer would be ``That's how you
become a professor.'' Some of them imagine themselves as unicorn startup
founders---their answer would be ``It will make you rich.''
Some of them don't even know who and where they want to be in a few
decades---their motive is yet to be discovered.
No matter who is in front of me, my first and the most
important goal as their teacher is to understand what they dream about
and then explain how learning the subject may help them get there.

\first{O}{nce} they know where they are going and why, they need
to know how to measure their progress.
They need some \textbf{metrics} of success.
They want to know whether the dream is getting closer or they are stuck.

``Just get a passing grade!'' is what some teachers use as a
scale of measurement. I see it differently. Honestly, I don't see myself
as most competent in the subjects I teach. In each of
them---be it object-oriented programming, software architecture, DevOps, or project management---there
are world-class experts sitting in program committees of international conferences and academic journals.
They, I believe, must be the judges of my students.
They are the best providers of knowledge and expertise the market has to offer.

My job, as a teacher, is to introduce my students to the market, which will
judge them, both while they study and then later, when they graduate and work
professionally. I have to explain and
demonstrate them how to submit their work to conferences, how to make
their code open source, how to present their work to funding committees,
how to publish their first blog posts, and so on.

I believe that the size of my classroom is too small for most talented students.
I have to introduce them to the rest of the world, which some of them will conquer, eventually.

\first{W}{hen} they have a reason to learn and the market, not me, appraises
them, I become their \textbf{peer}. Just like them, I try to publish
\href{https://www.yegor256.com/papers.html}{my work} at conferences,
I try to make \href{https://github.com/yegor256}{my software}
popular on GitHub, I try to learn new programming
languages, and try to get funding for my research projects.
We read the same books, watch the same YouTube videos, and attend the same webinars.
The only difference between us: they are much younger. They
think faster, they risk easier, and they love JavaScript.

Almost always, they need me as a walking addendum to Wikipedia:
when something is unclear, they expect me to explain. When I do it,
I see myself not as their teacher, but as just a bit more experienced colleague.
I don't teach them, I share what I know. Just like I share
\href{http://stackoverflow.com/users/187141/yegor256}{on Stack Overflow}
and on \href{https://www.yegor256.com}{my blog}.

That's how we learn together.

\end{document}
