\documentclass[12pt]{article}
\usepackage{xcolor}
  \definecolor{yb-blue}{RGB}{5,60,94}
\usepackage{hyperref}
  \hypersetup{colorlinks=true,urlcolor=yb-blue,pdfborder={0 0 0}}
\usepackage{graphicx}
\usepackage{microtype}
\usepackage[absolute]{textpos}
  \TPGrid{16}{16}
\usepackage[top=2\TPVertModule, bottom=2\TPVertModule, left=3\TPHorizModule, right=2\TPHorizModule]{geometry}
\usepackage{tikz}
\usepackage{fancyhdr}
  \pagestyle{fancy}
  \renewcommand{\headrulewidth}{0pt}
  \fancyhf{}
  \rhead{
    \begin{textblock}{1}[0,0](0,0){
      \tikz[x=\TPHorizModule,y=\TPVertModule] \filldraw[fill=yb-blue, draw=none] (0,0) rectangle (1,16);
    }\end{textblock}
  }
  \rfoot{\small Page~\thepage}

\begin{document}
\setlength{\topskip}{0mm}
\setlength{\parindent}{0pt}
\setlength{\parskip}{4pt}
\raggedright
\interfootnotelinepenalty=10000

\begin{textblock}{2.4}[1,0](14,2){
  \includegraphics[width=\textwidth]{../images/face-1200x1200.jpg}
}\end{textblock}

\section*{Yegor Bugayenko}

\href{mailto:yegor256@gmail.com}{yegor256@gmail.com}\\%
\href{https://www.yegor256.com}{www.yegor256.com}\\%
(408) 692-4742\\%
\href{https://github.com/yegor256}{GitHub} /
\href{https://stackexchange.com/users/63162/yegor256}{StackOverflow} /
\href{https://www.linkedin.com/in/yegor256}{LinkedIn} /
\href{https://twitter.com/intent/follow?screen_name=yegor256}{Twitter} /
\href{https://www.facebook.com/yegor256}{Facebook}

\subsection*{Seasoned Software Architect}

SCEA and PMP certified software developer and architect with
25+ years of engineering in complex projects and distributed software
development teams. Strong background in object-oriented analysis and design
with Java and C++. Took participation in over 150 successful projects,
including for IBM, Intel, and Expedia. \href{https://www.yegor256.com/pets.html}{Founder}
and \href{https://github.com/yegor256}{active contributor} of
a few open source projects. Author of
\href{https://www.yegor256.com/elegant-objects.html}{Elegant Objects}
book series about object-oriented programming,
regular \href{https://www.yegor256.com/talks.html}{speaker} at software conferences, regular software
\href{https://www.yegor256.com}{blogger}.

\subsection*{Hands-on Expertise}

\href{http://www.takes.org}{Takes} (web framework):
Java, XML/XSLT, HTTP, Maven, JUnit.

\href{http://www.jare.io}{Jare} (free instant CDN):
AWS CloudFront/DynamoDB, Java 8.

\href{http://www.rultor.com}{Rultor} (hosted DevOps assistant):
Chat bot, Docker, GitHub API, AWS EC2/S3.

\href{http://www.sixnines.io}{SixNines} (availability monitoring):
Ruby, Sinatra, Cucumber, XML/JSON.

\href{http://www.phprack.com}{phpRack} (integration testing framework):
PHP, jQuery, Phing, PHPUnit.

\href{https://github.com/yegor256/tacit}{Tacit} (CSS framework):
CSS, JavaScript, Grunt.

\subsection*{Recent Projects}

2018. \textbf{Zerocracy.com} (USA): AI-Powered Management Platform.
Designed and implemented a software platform for distributed developers;
Hosted on AWS with MongoDB, DynamoDB, S3, EC2;
The work is still in progress.

2018. \textbf{Zold.io} (USA): An experimental cryptocurrency for
fast micropayments, without Blockchain; it utilizes the proof-of-work
concept, but offers a completely unique solution to the decentralized
payment management problem.

2017. \textbf{Rultor.com} (USA): Continuous Delivery Cloud Platform (USA).
Designed and implemented a multi-node distributed build reactor;
AWS cloud hosting, via S3/EC2/CloudWatch;
Automated PhantomJS/CasperJS integration tests;
MongoDB, PostgreSQL, AWS DynamoDB/SQS/EC2/CF, CloudBees.

2013. \textbf{VakantieVeilingen.nl}: Auctioning Platform (The Netherlands).
Introduced unit testing, continuous delivery, and quality control techniques;
Automated pre-flight building mechanism;
Refactored Zend Framework based application with over 2mln hits per day;
PHP 5.3, PHPUnit, Ant, Jenkins, Vagrant, Chef, PhantomJS/CasperJS.

2012. \textbf{Netbout.com}: User Interface On-Demand.
Designed and developed a scalable solution on top of Amazon EBT/S3/CF/EC2;
Presented the product on a few VC forums in Silicon Valley;
HTML5/CSS3/jQuery, threads, Velocity, JUnit, Mockito, Facebook API, jQuery, MySQL.

2011. \textbf{Expinia.com}: Semantic Search Engine.
Designed and developed a search engine on top of Apache Lucene and Mallet;
Resolved performance and scalability issues (over 20M entities updated daily);
Implemented web layout in HTML5/CSS3;
Lucene 3.5, RDF, Mallet, RSS/Atom, RDF, H2 Database, MySQL, OAuth/LinkedIn.

\subsection*{Work Experience}

\textbf{\href{http://www.zerocracy.com}{Zerocracy, Inc.}} (Palo Alto, CA), 08/16--present.
CEO/Founder of an AI-powered platform for distributed software development teams.
Creating the chatbot and integrating it with GitHub, Slack, Telegram;
Organizing the pool of freelancers and customers;
Designing work \href{https://www.zerocracy.com/policy.html}{Policy} for the distributed community of developers;
Managing sales, marketing, and PR departments and activities.

\textbf{\href{http://www.technoparkcorp.com}{TechnoPark Corp.}} (Naples, FL), 05/00--08/16.
CEO/Co-Founder for Software development general contractor with offices in Florida and Eastern Europe.
Coordinating work of multiple multi-national software teams;
Overseeing management, design, programming, testing, and delivery;
Over 120 successful software projects for customers from North America and Europe;
Serving as point of contact for all strategic and technology governance-related issues;
Prepared and successfully passed ISO:9001 company certification;
Defined and incorporated CMMI Level 4 management standards;
Controlled implementation and patenting of an enterprise project management toolkit.

\textbf{\href{http://www.mika.ua}{MIKA Co.}} (Ukraine), 04/93--04/00.
Lead Software Developer and Technical Project Manager for Weighing and scaling equipment manufacturer.
Created and maintained project management and technical documentation;
Acted as liaison between all departments;
Created and reported daily status of projects;
Team leader on numerous projects utilizing Java and C++ platforms;
Created fully-integrated enterprise-wide information systems;
Performed requirements analysis, design, implementation, testing, and maintenance;
Hands-on programming in Java and C++.

\subsection*{Certifications}

Sun Certified Enterprise Architect (SCEA), 2010.

Zend Certified Engineer (ZCE) in PHP5, 2010.

Zend Framework Certified Engineer (ZFCE), 2010.

OMG Certified UML Professional (OCUP) Fundamental, 2011.

Project Management Professional (PMP), 2017.

IBM Certified Solution Designer for RUP, 2007.

PRINCE2 Foundation Certified, 2008.

Microsoft Certified Professional (MCP), Managing, Organizing, and Delivering by Using MSF 3.0, 2009.

CompTIA Project+ Certified, 2007.

IBM Certified Specialist for Requirements Management w/Use Cases, 2007.

COSMIC v3.0 Certified, 2008.

\subsection*{Education}

Master's Degree in Computer Science, 1993-1998\\
\href{http://dnu.dp.ua/}{Dnepropetrovsk National University}, Ukraine\\
Radio-Physics Department, Database Management Systems.

\subsection*{Public and Open Source Initiatives}

\href{http://www.takes.org}{Takes}: Open source Java web framework.

\href{http://www.cactoos.org}{Cactoos}: Object-oriented open source framework.

\href{http://www.jcabi.com}{JCabi}: Java AOP aspects, Maven plugins, objec wrappers, Java, 30K+ LOC.

\href{http://www.xembly.org}{Xembly}: XML manipulation language and library (Java and Ruby).

\href{http://www.qulice.com}{Qulice}: Java static code analysis toolkit, Java, 10K+ LOC.

\href{http://ieeexplore.ieee.org/document/6835311/}{IEEE P730} Software Quality Assurance Standard: volunteer.

\href{https://www.acm.org}{ACM}: Full member since April 1996.

\href{https://certification.pmi.org/registry.aspx}{PMI}: Member since January 2007.

\href{https://www.ieee.org}{IEEE}: Professional member since February 2007.

\subsection*{Books}

\href{https://amzn.to/2E5UHqZ}{Elegant Objects}, Volume~1, CreateSpace, 2016:
``There are 23 practical recommendations for object-oriented programmers. Most of them are completely
against everything you've read in other books.
For example, static methods, NULL references, getters, setters, and
mutable classes are called evil.''

\href{https://amzn.to/2J2s5T4}{Elegant Objects}, Volume~2, CreateSpace, 2017:
``Compound variable names, validators, private static literals, configurable objects, inheritance,
annotations, MVC, dependency injection containers, reflection, ORM and even algorithms are our enemies.''

\href{https://amzn.to/2u9BbqF}{Code Ahead}, Volume~1, CreateSpace, 2018:
``It's a semi-autobiographical fiction book about a software architect
who is involved in programming, debugging, releasing, testing,
organizing, team work, and management issues.''

\href{https://amzn.to/2GkuyXf}{256 Bloghacks}, CreateSpace, 2016:
``This book summarizes my experience of blogging for two and a half years and growing from zero to 60,000
unique visitors a month at www.yegor256.com; all dirty secrets revealed.''

\subsection*{Some recent conference talks}

\href{https://youtu.be/KCx1o_lSMkI}{Expertise vs Experts}\\
DevOpsDays, Boston, MA, USA, September 2018.

\href{https://youtu.be/cv23Z6xpwDw}{Java Annotations Are a Bad Idea}\\
JDK.IO, Copenhagen, Denmark, June 2017.

\href{https://www.youtube.com/watch?v=IGbteQpTNCA}{How Bright Is Your Future?}\\
RigaDevDays, Riga, Latvia, May 2017.

\href{https://www.youtube.com/watch?v=03PXmPc7Q3g}{ORM Is an Offensive Anti-Pattern}\\
\O{}redev 2016, Malm\"o, Sweden, November 2016.

\href{https://www.youtube.com/watch?v=3dJP_LtUGgg}{8 Maturity Levels of Continuous Integration}\\
DevOpsDays, Salt Lake City, UT, USA, June 2016.

\href{https://www.youtube.com/watch?v=nCGBgI1MNwE}{Need It Robust? Make It Fragile}\\
DEVit, Thessaloniki, Greece, May 2016.

\href{https://www.youtube.com/watch?v=LB_YLWhGrco}{Meetings and Motivation, Friends or Enemies?}\\
SEDC, Washington, D.C., USA, March 2016.

\href{https://www.youtube.com/watch?v=qRZYJGYdrwk}{XDSD: Meetings-Free Software Development Methodology}\\
The Entrepreneurs' Club, Palo Alto, CA, USA, February 2016.

\subsection*{Patents, Patent Applications}

\href{https://patents.google.com/patent/US20120117164}{12/943,022}
``Method and Software of NetBout''

\href{https://patents.google.com/patent/US20120023476}{12/840,306}
``Puzzle Driven Development (PDD) Method and Software''

\href{https://patents.google.com/patent/US20110196798}{12/703,202}
``Project Management Robot Method and Software''

\href{https://patents.google.com/patent/US20100114638}{12/264,370}
``Method and software for the measurement of quality of process''

\href{https://patents.google.com/patent/US20100042968}{12/193,010}
``Method for software cost estimating using scope champions''

\subsection*{Publications}

\emph{\href{https://www.yegor256.com/pdf/2018/discovering-bugs.pdf}{Discovering Bugs, or Ensuring Success?}}\\
Communications of the ACM, Volume 61, Number 9, Aug 2018.

\emph{\href{https://www.yegor256.com/pdf/2018/we-are-done-with-hacking.pdf}{We are Done with ``Hacking,''}}\\
Communications of the ACM, Volume 61, Number 7, Jul 2018.

\emph{\href{https://www.yegor256.com/pdf/2010/phpArchitect-conflicts.pdf}{How to Prevent SVN Conflicts in Distributed Agile PHP Projects}},\\
php$|$Architect, Aug 2010,

\emph{\href{https://www.yegor256.com/pdf/2010/phpArchitect-phpRack.pdf}{phpRack---Integration Testing Framework}},\\
php$|$Architect, Jun 2010,

\emph{\href{https://www.yegor256.com/pdf/2010/phpArchitect-fazend-orm.pdf}{FaZend Object Relational Mapping}},\\
php$|$Architect, Feb 2010,

\emph{\href{https://www.yegor256.com/pdf/2009/IWSM09-article.pdf}{Quality of Process Control in Software Projects}},\\
IWSM/Mensura, Amsterdam, Nov 2009,

\emph{Quality of Code Can be Planned and Controlled},\\
VALID, Porto, Portugal, Sep 2009,

\emph{\href{https://www.yegor256.com/pdf/2009/SEAFOOD09-article.pdf}{Competitive Risk Identification Method for Distributed Teams}},\\
SEAFOOD, Zurich, Jul 2009,

\emph{Method for Software Cost Estimating w/Scope Champions},\\
PROFES, Oulu, Finland, Jun 2009.

\emph{\href{https://www.yegor256.com/pdf/1998/KRDB98-article.pdf}{The Interactive Databases Approach to the User Interface Modeling}},\\
KRDB, Seattle, May 1998.

\end{document}
